\documentclass[12 pt]{exam}
\usepackage{amsmath}
\usepackage{amssymb}
\usepackage[utf8]{inputenc}\usepackage[margin=2cm]{geometry}

\pointpoints{Punkt}{Punkte} 
\qformat{{\bf Aufgabe \thequestion.}\ (\thepoints)\hfill} 
\hpword{Punkte} 
\hsword{Erreicht} 
\hqword{Aufgabe} 

\setlength{\parindent}{0cm}
\setlength{\parskip}{0cm}
\renewcommand{\baselinestretch}{1.5}


 
\begin{document} 
 
\noindent Name: \hrulefill \qquad \qquad \qquad \qquad \qquad \qquad Datum: 08.01.2019\\ \\ 
\begin{center} 
%{\Huge Test}\\ 
\vspace{0.75cm}
\gradetablestretch{1.25}
\addpoints 
\gradetable[h][questions] 
\end{center} 
\vspace{0.75cm} 
\begin{questions} 
	
\question[2] Zwei Vielecke sind \"ahnlich, wenn die folgenden Bedingungen gelten
\begin{itemize}
	\item[$\bullet$] ..............................................................................
	\item[$\bullet$] ..............................................................................
\end{itemize}
 
\question[2] Zwei \"ahnliche Dreiecke haben den Umfang von $5\ cm$ bzw. $20\ cm$. \\
Die Seite $a = 1\ cm$ des urspr\"unglichen Dreiecks entspricht einer Seite  $a' = ....... \ cm$ des Bilddreiecks.\\
Die Seite $b' = 8\ cm$ des Bilddreiecks entspricht einer Seite $b =....... \ cm$ des ursp\"unglichen Dreiecks.

\question[1] Zwei \"ahnliche Vielecke mit Fl\"acheninhaltsverh\"altnis $\frac{9}{25}$ haben 
\begin{itemize}
	\item[$\bullet$] den Streckfaktor ........;
	\item[$\bullet$] das L\"angenverh\"altnis ......... . 
\end{itemize}
\end{questions}


\end{document}