% !TeX document-id = {44d9ad48-b5ed-4d9a-bfde-b8acb6278d5d}
% !TeX TXS-program:compile = txs:///dvi-ps-pdf-chain

\documentclass[a4paper,11pt]{exam}

%%%%%%%%%%%%%%%%%%%%%%%%%%%%%%%%%%%%%%%%%%%%%%%%%%%%%%%%%%%%%%%%%%%%%%%%%%%
\printanswers
%%%%%%%%%%%%%%%%%%%%%%%%%%%%%%%%%%%%%%%%%%%%%%%%%%%%%%%%%%%%%%%%%%%%%%%%%%%

\usepackage{amssymb, amsfonts, amsmath}
\usepackage{pstricks-add}
\usepackage[margin=2.4cm]{geometry}
\usepackage{enumitem}
\usepackage{color}
\usepackage{mdframed}
\usepackage[colorlinks=true,
	linkcolor=black,urlcolor=black,citecolor=black,
]{hyperref}
\usepackage[utf8]{inputenc}

\colorgrids
\setlength{\gridlinewidth}{0.1pt}	% default is 0.1pt

\newcommand*{\lightgraygrid}{
	\definecolor{GridColor}{rgb}{0.85, 0.85, 0.85}
	\setlength{\gridsize}{4mm}
}
\newcommand*{\graygrid}{
	\definecolor{GridColor}{rgb}{0.7, 0.7, 0.7}
	\setlength{\gridsize}{4mm}
}

\def\gridtext#1#2{
	\leavevmode\rlap{
		\vbox to #1{\fillwithgrid{#1}}}
		{\hspace{0.2cm}\vbox to #1{\vspace*{-0.2cm}\vfil#2\vfil}
	}% rlap
}


\setlength\answerclearance{0.5ex}
\CorrectChoiceEmphasis{\rmfamily}

\newcommand*{\graylinefill}{\textcolor[RGB]{200,200,200}\hrulefill}
\newcommand*{\grayfillin}[2][2cm]{
	\textcolor[RGB]{200,200,200}{
		\raisebox{0pt}[15pt][0pt]{\null}\fillin[\black #2][#1]
	}
}

\footer{}{Seite \thepage\ von \numpages}{}

\setlength{\parskip}{.6\baselineskip}
\parindent=0pt
%\linespread{1.1}

% enumeration
\setlist[itemize]{topsep=3pt}
\setlist[enumerate]{topsep=3pt}

%%%%%%%%%%%%%%%%%%%%%%%%%%%%%%%%%%%%%%%%%%%%%%%%%%%%%%%%%%%%%%%%%%%%%%%%%%%%
\begin{document}
%%%%%%%%%%%%%%%%%%%%%%%%%%%%%%%%%%%%%%%%%%%%%%%%%%%%%%%%%%%%%%%%%%%%%%%%%%%%

\ifprintanswers \lightgraygrid{} \else \graygrid{} \fi

\begin{center}
	\ifprintanswers {\large \textbf{\red Lösungen}}\\[2cm] \else
	\vspace*{1.7cm} \fi
	\LARGE{Lineare Programmierung / Optimierung}\\
	\vspace{1cm}
	\large{Kapitel 10.3}\\
	\vspace{.5cm}
	\large{29. November 2018}\\
\end{center}

\vfil

\section*{Eine Maximalaufgabe}

\begin{mdframed}
Sie sind Mitglied der AMP (Anti-Mathe-Partei), und möchten mittels einer Volksinitiative in der Verfassung festlegen, dass an Schulen keine Mathematik mehr unterrichtet werden darf.

Um \textbf{Unterschriften zu sammeln}, haben Sie zwei Möglichkeiten zur Verfügung: Sie können entlang der Strasse grosse \textbf{Plakate} aufhängen, oder an öffentlichen Orten \textbf{Flyer} verteilen. Es hat sich gezeigt, dass die Initiative auf riesige Zustimmung stösst: Pro Plakat werden durchschnittlich 500 Personen erreicht, die sich dazu entscheiden eine Unterschrift zu geben, und jedes Dutzend verteilte Flyer führt zu 100 Unterschriften (Mund-zu-Mund Propaganda etc.)!

Leider gibt es einige \textbf{Einschränkungen}:
\begin{itemize}
\item \textit{Budget}: Sie haben ein kleines Budget von nur 7'000 Franken zur Verfügung. Der Druck eines Plakates kostet 30 Franken, und der Druck eines Dutzend Flyer kostet 10 Franken.
\item \textit{Zeit}: Um ein Plakat zu drucken \& aufzuhängen müssen Sie mit 2h Arbeit rechnen. Für die Herstellung und Verteilung von einem Dutzend Flyer wird ca. eine Viertel-Stunde gebraucht. Leider haben Sie zeitlich nur insgesamt 250 Stunden zur Verfügung.
\item \textit{Logistik}: Aus Platzgründen können maximal 90 Plakate aufgehängt werden.
\end{itemize}

{Frage}: Wie viele Plakate und Flyer müssen Sie in Umlauf bringen, um \textbf{möglichst viele Unterschriften} zu erreichen? Schaffen Sie es mit den Ihnen zur Verfügung stehenden Mitteln, die für eine Volksinitiative nötigen 100'000 Unterschriften zu sammeln?
\end{mdframed}

\newpage

\section*{Lösung
\ifprintanswers {(ausgefüllt)}
\fi}

\renewcommand*\thesubsection{\arabic{subsection}.}

\subsection{Festlegen der Variablen}

$x_1=$
\ifprintanswers Anzahl Plakate \fi
\graylinefill\medskip

$x_2=$ \ifprintanswers Anzahl Flyer (in Dutzend) \fi
\graylinefill\medskip

\subsection{Bestimmen des Ungleichungssystems}

Einschränkungen durch das Budget:\\
\gridtext{2cm}{
	$(1)\quad$ \ifprintanswers $30x_1+10x_2\leq 7'000 \quad
	\iff \quad x_2\leq -3x_1+700$ \fi\vfil\vfil
}

Zeitliche Beschränkung:\\
\gridtext{2cm}{
	$(2)\quad$ \ifprintanswers $2x_1+0.25x_2\leq 250 \quad
	\iff \quad x_2 \leq -8x_1 + 1000$ \fi\vfil\vfil
}

Platzbeschränkung für Plakate:\\
\gridtext{1.5cm}{
	$(3)\quad$ \ifprintanswers $x_1\leq 90$ \fi
}

Nichtnegativitäts-Bedingungen:\\
\gridtext{1.5cm}{
	$(4)\quad$ \ifprintanswers $x_1\geq 0$ \fi
	\hfill
	$(5)\quad$ \ifprintanswers $x_2\geq 0$ \fi
	\hfill\null
}

\subsection{Zeichnen des Ungeichungssystems}\label{koordinatensystem}

Alle fünf Ungleichungen müssen gleichzeitig erfüllt sein. Die Lösungsmenge dieses Ungleichungssystems besteht aus allen Punkten $(x_1,x_2)$, die innerhalb des durch diese Ungleichungen festgelegten Bereiches liegen.

Zum Beispiel ist der Punkt $(10,700)$ nicht zulässig: Wenn wir $10$ Plakate und $700$ Dutzend Flyer herstellen, übersteigt dies \grayfillin[4cm]{das Budget}.


\psset{xunit=.12cm,yunit=.012cm,algebraic=true,dimen=middle,dotstyle=o,dotsize=5pt 0,linewidth=1pt,arrowsize=3pt 2,arrowinset=0.25}
\begin{pspicture*}(-25.,-250.)(111.,910.)
\multips(0,-240)(0,20.0){60}{\psline[linestyle=dashed,linecap=1,dash=1.5pt 1.5pt,linewidth=0.3pt,linecolor=lightgray]{c-c}(-25.,0)(110.,0)}
\multips(0,-200)(0,100.0){12}{\psline[linestyle=solid,linecap=1,linewidth=0.5pt,linecolor=lightgray]{c-c}(-25.,0)(110.,0)}

\multips(-24,0)(2.0,0){70}{\psline[linestyle=dashed,linecap=1,dash=1.5pt 1.5pt,linewidth=0.3pt,linecolor=lightgray]{c-c}(0,-250.)(0,900.)}
\multips(-20,0)(10.0,0){14}{\psline[linestyle=solid,linecap=1,linewidth=0.5pt,linecolor=lightgray]{c-c}(0,-250.)(0,900.)}

\psaxes[labelFontSize=\scriptstyle,xAxis=true,yAxis=true,Dx=10.,Dy=100.,ticksize=-2pt 0,subticks=2]{->}(0,0)(-25.,-250.)(105.,850.)

\rput[cl](108,0){\textbf{$x_1$}}
\rput[bc](0,870){\textbf{$x_2$}}

\ifprintanswers

	\newrgbcolor{darkgreen}{0. 0.6 0.}
	\newrgbcolor{aqaqaq}{0.45 0.45 0.45}
	
	\pspolygon[fillcolor=aqaqaq,fillstyle=solid,opacity=0.25](0,0)(90,0)(90,280)(60,520)(0,700)
	
	\psplot[linewidth=2.pt,linecolor=red]{-41.618524295915094}{161.2791520821792}{(-0.-50.*x)/10.}
	\psline[linewidth=1.2pt](90.,-245.77859789160362)(90.,913.071675128534)
	\psplot[linewidth=1.2pt,linecolor=blue]{-41.618524295915094}{161.2791520821792}{(--250.-2.*x)/0.25}
	\psplot[linewidth=1.2pt,linecolor=darkgreen]{-41.618524295915094}{161.2791520821792}{(--7000.-30.*x)/10.}
	\psplot[linewidth=2.pt,linecolor=red]{-41.618524295915094}{161.2791520821792}{(--8200.-50.*x)/10.}
	
	\rput[tl](35,-130){\red{Zielfunktion durch den Nullpunkt}}
	\rput[tl](32.34222713235709,342.2159661208894){$\mathbb{L}(S)$}
	\rput[tl](100,290){$\red{Z}$}
	\rput[tl](10,650){$\darkgreen{(1)}$}
	\rput[tl](37,750){$\blue{(2)}$}
	\rput[tl](85,800){$(3)$}
	
	\psdots[dotstyle=*](60.,520.)

\fi
\end{pspicture*}


\subsection{Aufstellen der Zielfunktion}

Die Zielfunktion bezieht sich auf diejenige Grösse, die aufgrund der Aufgabenstellung optimiert werden soll, also in unserem Fall die \grayfillin[6cm]{Anzahl Unterschriften}, die möglichst gross werden muss.

Die Zielfunktion ist abhängig von
\grayfillin[1.5cm]{$x_1$} (Anzahl aufgehängter Plakate) und
\grayfillin[1.5cm]{$x_2$} (Anzahl verteilter Flyer, in Dutzend):\\
\gridtext{2cm}{
	$Z= $ \ifprintanswers $500x_1+100x_2$ \fi
}

\newpage
\subsection{Zeichnen der Zielfunktion durch den Nullpunkt}
Die Zielfunktion (Anzahl Unterschriften) soll möglichst gross werden. Um herauszufinden, für welche Werte von $x_1$ und $x_2$ dies der Fall ist, lösen wir die Gleichung der Zielfunktion nach $x_2$ auf:\\

\gridtext{2.5cm}{
	\ifprintanswers $x_2=-5x_1 + \dfrac{Z}{100}$\fi
}\\

Dies ist die Gleichung einer Geraden mit Achsabstand \grayfillin[1.5cm]{$\frac{Z}{100}$}. Da $Z$ zunächst noch unbekannt ist, zeichnet man diese Gerade vorläufig mal an einer beliebigen Stelle ein, am einfachsten mit Achsabschnitt $0$ (also durch den Ursprung
\ifprintanswers -- siehe unter~\ref{koordinatensystem})\else \hspace{-.3em})\fi.

\subsection{Bestimmen des optimalen Punktes und Berechnen der zugehörigen Koordinaten}

Was geschieht, wenn die eben eingezeichnete Gerade nach oben verschoben wird? Offensichtlich wird so 
\grayfillin[6cm]{der Achsabstand}
immer grösser, und damit auch
\grayfillin[2cm]{$Z$}.\medskip

{\leftskip=1cm\textbf{Beispiel:} Wenn die Gerade so weit nach oben verschoben wird, dass der Achsabstand $\frac{Z}{100}=400$ beträgt, dann ist $Z=$\grayfillin[2cm]{$40'000$}. Das heisst, für alle Punkte $(x_1,x_2)$ auf dieser Geraden erreichen wir $40'000$ Unterschriften!\par}\medskip

Um die Zielfunktion zu maximieren, muss sie also möglichst weit nach oben verschoben werden, und zwar so weit, dass sie gerade noch in der Lösungsmenge $\mathbb{L}(S)$ liegt! Dies liefert den optimalen \ifprintanswers Punkt -- siehe unter~\ref{koordinatensystem}\else Punkt.\fi

Die \textbf{Koordinaten des optimalen Punktes} ergeben sich als Schnittpunkt der Geraden
\begin{center}
\grayfillin[7cm]{$\darkgreen (1)\quad x_2=-3x_1+700$} und
\grayfillin[7cm]{$\blue (2)\quad x_2=-8x_1+1000$}:
\end{center}

\gridtext{4cm}{
	\ifprintanswers
	$\darkgreen -3x_1+700 \black = \blue -8x_1+1000 \black
	\quad \iff \quad 5x_1 = 300 \quad \iff \quad x_1=60.$
	\vfil
	$x_2=\darkgreen -3\cdot \black 60 \darkgreen + 700 \black = 520$.
	\fi
}\\

Der optimale Punkt liegt bei
$P_{\text{max}}($
\ifprintanswers $60$ \else \hspace{1cm} \fi,
\ifprintanswers $520$ \else \hspace{1cm} \fi$)$. Das heisst, wir erreichen am meisten Unterschriften, wenn wir \grayfillin[2cm]{$60$} Plakate aufhängen und \grayfillin[2cm]{$520$} Dutzend Flyer verteilen.


\subsection{Berechnen des Wertes der Zielfunktion beim optimalen Punkt}

Jetzt kommt die Stunde der Wahrheit: Erreichen wir die nötigen 100'000 Unterschriften? Dazu muss der Wert der Zielfunktion beim optimalen Punkt berechnet werden:\\
\gridtext{2cm}{
	\ifprintanswers
$Z=500\cdot 60 + 100\cdot 520 = 82'000$.
\fi}\\


\checkboxchar{$\Box$}
\checkedchar{$\blacksquare$}

100'000 Unterschriften erreicht?

\begin{oneparcheckboxes}
\choice Nein (leider)
\correctchoice Nein (zum Glück)
\choice Ja (leider)
\choice Ja (zum Glück).
\end{oneparcheckboxes}

\vfill

\hrule
\subsection*{Bemerkungen}
\begin{itemize}
	\item Überlegen Sie sich immer zuerst, ob die Zielfunktion maximiert (Maximalaufgaben) oder minimiert (Minimalaufgaben) werden muss. Anschliessend müssen Sie sich überlegen, ob die Gerade dazu nach oben oder nach unter verschoben werden muss.
	
	\item Wenn die Zielfunktion nach $x_2$ aufgelöst wurde, hat sie in der Regel die Form
	\begin{align*}
	x_2&=-m\cdot x_1 + \boxed{\text{ein Term mit Z}}.
	\intertext{In den Lösungen im Buch wird das geschrieben als}
	f_Z(x_1)&=-m\cdot x_1 + \boxed{c},
	\end{align*}
	da $Z$ (und damit der Achsabstand) ja noch unbekannt ist.
	
	\item Besuchen Sie unsere Website: \url{https://amp-initiative.netlify.com}.
	
\end{itemize}

\end{document}